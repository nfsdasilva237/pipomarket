\documentclass[12pt,a4paper]{article}

% Packages
\usepackage[utf8]{inputenc}
\usepackage[french]{babel}
\usepackage[T1]{fontenc}
\usepackage{geometry}
\geometry{a4paper, margin=2.5cm}
\usepackage{hyperref}
\usepackage{listings}
\usepackage{xcolor}
\usepackage{graphicx}
\usepackage{fancyhdr}
\usepackage{tcolorbox}
\usepackage{enumitem}
\usepackage{longtable}
\usepackage{booktabs}

% Configuration des couleurs
\definecolor{codegreen}{rgb}{0,0.6,0}
\definecolor{codegray}{rgb}{0.5,0.5,0.5}
\definecolor{codepurple}{rgb}{0.58,0,0.82}
\definecolor{backcolour}{rgb}{0.95,0.95,0.92}

% Style pour le code
\lstdefinestyle{mystyle}{
    backgroundcolor=\color{backcolour},
    commentstyle=\color{codegreen},
    keywordstyle=\color{magenta},
    numberstyle=\tiny\color{codegray},
    stringstyle=\color{codepurple},
    basicstyle=\ttfamily\footnotesize,
    breakatwhitespace=false,
    breaklines=true,
    captionpos=b,
    keepspaces=true,
    numbers=left,
    numbersep=5pt,
    showspaces=false,
    showstringspaces=false,
    showtabs=false,
    tabsize=2
}
\lstset{style=mystyle}

% Configuration hyperref
\hypersetup{
    colorlinks=true,
    linkcolor=blue,
    filecolor=magenta,
    urlcolor=cyan,
    pdftitle={Guide Publication PipoMarket},
    pdfauthor={PipoMarket Team},
}

% En-tête et pied de page
\pagestyle{fancy}
\fancyhf{}
\rhead{Guide Publication PipoMarket}
\lhead{\leftmark}
\cfoot{\thepage}

% Titre
\title{\Huge\textbf{Guide Complet de Publication} \\ \Large Play Store \& App Store \\ \vspace{0.5cm} \large PipoMarket}
\author{Documentation Technique}
\date{\today}

\begin{document}

\maketitle
\newpage

\tableofcontents
\newpage

% =====================================================
% PARTIE 1: QUICK START
% =====================================================

\section{Quick Start - Démarrage Rapide}

\begin{tcolorbox}[colback=yellow!10!white,colframe=yellow!75!black,title=⚡ Pour Commencer Demain Matin]
Ce guide vous permet de publier PipoMarket sur Play Store et App Store en une journée de travail.
\end{tcolorbox}

\subsection{Vue d'ensemble rapide}

\begin{table}[h]
\centering
\begin{tabular}{|l|l|l|l|l|}
\hline
\textbf{Plateforme} & \textbf{Build} & \textbf{Review} & \textbf{Coût} & \textbf{Difficulté} \\
\hline
Play Store & 20-30 min & 3-7 jours & 25\$ (unique) & ⭐⭐ Facile \\
App Store & 20-30 min & 1-2 semaines & 99\$/an & ⭐⭐⭐ Moyen \\
\hline
\end{tabular}
\caption{Comparaison des plateformes}
\end{table}

\subsection{Plan de la journée}

\subsubsection{1. MATIN (1-2h) - Créer les comptes}

\textbf{Play Store:}
\begin{itemize}
    \item URL: \url{https://play.google.com/console/signup}
    \item Coût: 25\$ (carte bancaire)
    \item Activation: Immédiate ✅
\end{itemize}

\textbf{App Store:}
\begin{itemize}
    \item URL: \url{https://developer.apple.com/programs/enroll/}
    \item Coût: 99\$/an (carte bancaire)
    \item Activation: 24-48h ⏱️
\end{itemize}

\subsubsection{2. MIDI (2-3h) - Préparer les assets}

\textbf{A. Screenshots (Priorité 1)}

Sur votre téléphone Android:
\begin{enumerate}
    \item Lancez l'app: \texttt{npx expo start}
    \item Scannez le QR code avec Expo Go
    \item Prenez 4-6 screenshots:
    \begin{itemize}
        \item Écran d'accueil
        \item Page produit
        \item Panier
        \item Profil startup
    \end{itemize}
    \item Sauvegardez dans \texttt{screenshots/}
\end{enumerate}

\textbf{B. Feature Graphic (Play Store)}
\begin{itemize}
    \item Dimensions: \textbf{1024 × 500 px}
    \item Format: PNG ou JPG
    \item Outil recommandé: Canva.com (gratuit)
    \item Contenu: Logo + "PipoMarket - Marketplace Camerounaise"
\end{itemize}

\textbf{C. Politique de confidentialité}
\begin{itemize}
    \item Copiez le template (voir section Templates)
    \item Mettez-le en ligne sur: \texttt{pipomarket.com/privacy}
\end{itemize}

\subsubsection{3. APRÈS-MIDI (1h) - Build Android}

\begin{lstlisting}[language=bash,caption={Commandes Build Android}]
# Installation EAS CLI
npm install -g eas-cli

# Connexion
eas login

# Initialisation
eas init

# Build Android
eas build --platform android --profile production
\end{lstlisting}

⏱️ \textbf{Durée du build}: 20-30 minutes (automatique dans le cloud)

Pendant l'attente, passez à l'étape suivante.

\subsubsection{4. PENDANT LE BUILD - Préparer Play Store}

\begin{enumerate}
    \item Allez sur: \url{https://play.google.com/console}
    \item Cliquez "Créer une application"
    \item Nom: \textbf{PipoMarket}
    \item Type: Application, Gratuite
    \item Préparez:
    \begin{itemize}
        \item Feature Graphic (1024×500)
        \item Screenshots (minimum 2)
        \item Icône: \texttt{assets/images/icon.png}
        \item Descriptions (voir section Templates)
    \end{itemize}
\end{enumerate}

\subsubsection{5. FIN D'APRÈS-MIDI - Soumission Play Store}

\begin{enumerate}
    \item Build Android terminé → Téléchargez le fichier \texttt{.aab}
    \item Play Console → Production → Créer une version
    \item Uploadez le \texttt{.aab}
    \item Uploadez Feature Graphic + Screenshots
    \item Remplissez tous les champs (descriptions, contact, etc.)
    \item Cliquez \textbf{"Déployer en production"}
\end{enumerate}

✅ \textbf{FAIT!} Attente: 3-7 jours pour la review Google.

\subsubsection{6. SOIR (1h) - Build iOS}

\textbf{Seulement si le compte Apple est activé}

\begin{lstlisting}[language=bash,caption={Build iOS}]
eas build --platform ios --profile production
\end{lstlisting}

Pendant le build (20-30 min):
\begin{enumerate}
    \item Allez sur: \url{https://appstoreconnect.apple.com}
    \item Créez l'app: PipoMarket
    \item Bundle ID: \texttt{com.pipomarket.app}
\end{enumerate}

\subsubsection{7. PLUS TARD - Soumission App Store}

Après que le build iOS apparaisse dans App Store Connect:
\begin{enumerate}
    \item Uploadez screenshots iPhone (voir dimensions exactes plus loin)
    \item Copiez-collez les textes (section Templates)
    \item Remplissez tous les champs
    \item Cliquez \textbf{"Envoyer pour examen"}
\end{enumerate}

✅ \textbf{FAIT!} Attente: 1-2 semaines pour la review Apple.

\subsection{Checklist rapide}

\subsubsection{Avant de dormir ce soir}
\begin{itemize}[label=$\square$]
    \item Lire ce document en entier
    \item Préparer CB pour payer les comptes (25\$ + 99\$)
\end{itemize}

\subsubsection{Demain matin (7h-9h)}
\begin{itemize}[label=$\square$]
    \item Créer compte Play Store (25\$)
    \item Créer compte Apple Developer (99\$)
    \item Café ☕
\end{itemize}

\subsubsection{Demain midi (12h-15h)}
\begin{itemize}[label=$\square$]
    \item Prendre screenshots sur téléphone
    \item Créer Feature Graphic sur Canva
    \item Copier politique confidentialité sur site
    \item Déjeuner 🍽️
\end{itemize}

\subsubsection{Demain après-midi (15h-18h)}
\begin{itemize}[label=$\square$]
    \item Installer EAS: \texttt{npm install -g eas-cli}
    \item Build Android: \texttt{eas build --platform android}
    \item Créer app sur Play Console
    \item Remplir fiche Play Store
    \item Télécharger .aab et soumettre
    \item Build iOS: \texttt{eas build --platform ios}
\end{itemize}

\subsubsection{Demain soir (19h-20h)}
\begin{itemize}[label=$\square$]
    \item Créer app sur App Store Connect
    \item Préparer screenshots iPhone
    \item Repos! 😴
\end{itemize}

\subsection{Budget et Timeline}

\begin{table}[h]
\centering
\begin{tabular}{|l|l|}
\hline
\textbf{Item} & \textbf{Prix} \\
\hline
Play Store & 25\$ (une fois) \\
App Store & 99\$ (par an) \\
\textbf{TOTAL} & \textbf{124\$} \\
\hline
\end{tabular}
\caption{Budget total}
\end{table}

\begin{table}[h]
\centering
\begin{tabular}{|l|l|l|}
\hline
\textbf{Action} & \textbf{Durée travail} & \textbf{Durée attente} \\
\hline
Créer comptes & 30 min & - \\
Préparer assets & 2-3h & - \\
Builds (auto) & 1h & - \\
Remplir fiches & 2h & - \\
\textbf{Total travail} & \textbf{~6h} & - \\
\hline
Review Play Store & - & 3-7 jours \\
Review App Store & - & 1-2 semaines \\
\hline
\end{tabular}
\caption{Timeline complète}
\end{table}

\newpage

% =====================================================
% PARTIE 2: GUIDE DÉTAILLÉ
% =====================================================

\section{Guide Détaillé - Action Stores}

\subsection{Préparation}

\subsubsection{Créer les comptes développeurs}

\textbf{Play Store (Android)}
\begin{enumerate}
    \item Allez sur: \url{https://play.google.com/console/signup}
    \item Payez 25\$ (frais unique, à vie)
    \item Remplissez informations développeur
    \item Activation: immédiate
\end{enumerate}

\textbf{App Store (iOS)}
\begin{enumerate}
    \item Allez sur: \url{https://developer.apple.com/programs/enroll/}
    \item Payez 99\$/an
    \item Remplissez informations développeur
    \item Activation: 24-48 heures (parfois plus long)
\end{enumerate}

\subsubsection{Installation EAS CLI}

Ouvrez votre terminal Windows et exécutez:

\begin{lstlisting}[language=bash]
npm install -g eas-cli
\end{lstlisting}

Vérifiez l'installation:
\begin{lstlisting}[language=bash]
eas --version
\end{lstlisting}

\subsection{Build Android (Play Store)}

\subsubsection{Connexion et initialisation}

\begin{lstlisting}[language=bash]
# 1. Se connecter a Expo
eas login

# 2. Initialiser le projet EAS
eas init

# 3. Configurer le projet
eas build:configure
\end{lstlisting}

Quand il demande:
\begin{itemize}
    \item \textbf{"Select platform"}: Choisissez \texttt{All} (pour préparer iOS aussi)
\end{itemize}

\subsubsection{Vérifier app.json}

Assurez-vous que ces informations sont correctes dans \texttt{app.json}:

\begin{lstlisting}[language=json]
{
  "expo": {
    "name": "PipoMarket",
    "version": "1.0.0",
    "android": {
      "package": "com.pipomarket.app",
      "versionCode": 1
    }
  }
}
\end{lstlisting}

\subsubsection{Lancer le build Android}

\begin{lstlisting}[language=bash]
eas build --platform android --profile production
\end{lstlisting}

\textbf{Ce qui va se passer:}
\begin{enumerate}
    \item Upload de votre code vers Expo
    \item Build dans le cloud (20-30 minutes)
    \item Vous recevrez un email quand c'est prêt
    \item Vous pourrez télécharger le fichier \texttt{.aab}
\end{enumerate}

\textbf{Pendant l'attente}, préparez votre fiche Play Store.

\subsection{Préparer la fiche Play Store}

\subsubsection{Assets nécessaires}

\textbf{1. Feature Graphic (OBLIGATOIRE)}
\begin{itemize}
    \item Dimensions: \textbf{1024 × 500 px}
    \item Format: PNG ou JPG
    \item Contenu: Logo + Texte "PipoMarket - Marketplace Camerounaise"
    \item Outil: Canva, Photoshop, ou GIMP
\end{itemize}

\textbf{2. Screenshots Android (MINIMUM 2)}
\begin{itemize}
    \item Dimensions recommandées: \textbf{1080 × 1920 px}
    \item Minimum: 2 screenshots
    \item Recommandé: 4-8 screenshots
    \item Contenu suggéré:
    \begin{itemize}
        \item Écran d'accueil avec produits
        \item Page produit
        \item Panier
        \item Profil startup
        \item Commandes
    \end{itemize}
\end{itemize}

\textbf{Comment prendre des screenshots:}
\begin{lstlisting}[language=bash]
# Lance l'app en emulateur
npx expo run:android

# Dans l'emulateur, utilisez les boutons de capture
# Ou utilisez votre telephone Android
\end{lstlisting}

\textbf{3. Icône app}
\begin{itemize}
    \item Déjà prêt: \texttt{assets/images/icon.png} (1024×1024)
\end{itemize}

\subsubsection{Textes à préparer}

\textbf{Description courte (80 caractères max)}

\textit{Marketplace camerounaise - Découvrez et soutenez les startups locales}

\textbf{Catégorie}: Shopping

\textbf{Type de contenu}: Pour tous publics

\textbf{E-mail de contact}: Votre email

\textbf{Politique de confidentialité}: URL obligatoire (voir section suivante)

\subsection{Politique de confidentialité (OBLIGATOIRE)}

Vous DEVEZ avoir une URL de politique de confidentialité accessible publiquement.

\textbf{Option A}: Créer une page sur votre site \texttt{pipomarket.com/privacy}

\textbf{Option B}: Utiliser un générateur en ligne
\begin{itemize}
    \item \url{https://www.privacypolicygenerator.info/}
    \item \url{https://app-privacy-policy-generator.firebaseapp.com/}
\end{itemize}

Un template complet est fourni dans la section Templates de ce document.

\subsection{Soumission sur Play Store}

\subsubsection{Télécharger le build}

\begin{enumerate}
    \item Quand le build est prêt, allez sur: \\ \url{https://expo.dev/accounts/[compte]/projects/pipomarket/builds}
    \item Cliquez sur le build Android
    \item Téléchargez le fichier \texttt{.aab}
\end{enumerate}

\subsubsection{Upload sur Play Console}

\begin{enumerate}
    \item Allez sur: \url{https://play.google.com/console}
    \item Cliquez \textbf{"Créer une application"}
    \item Remplissez:
    \begin{itemize}
        \item Nom: \textbf{PipoMarket}
        \item Langue par défaut: \textbf{Français}
        \item Type: \textbf{Application}
        \item Gratuite/Payante: \textbf{Gratuite}
    \end{itemize}
    \item Section "Production" → "Créer une version"
    \item Uploadez le fichier \texttt{.aab}
    \item Remplissez tous les champs:
    \begin{itemize}
        \item Feature graphic
        \item Screenshots
        \item Description courte
        \item Description complète
        \item Icône
        \item Catégorie
        \item Email de contact
        \item Politique de confidentialité
    \end{itemize}
    \item \textbf{Questionnaire de contenu}:
    \begin{itemize}
        \item Pas de contenu pour adultes: Non
        \item Annonces: Non
        \item Collecte de données: Oui (email, nom, téléphone)
    \end{itemize}
    \item Tarification: Gratuite dans tous les pays
    \item Cliquez "Vérifier la version"
    \item Cliquez "Déployer en production"
\end{enumerate}

\subsubsection{Attente de la review}

\begin{itemize}
    \item ⏱️ \textbf{3 à 7 jours} en moyenne
    \item Vous recevrez un email de Google
    \item Statut visible dans Play Console
\end{itemize}

\subsection{Build iOS (App Store)}

\subsubsection{Prérequis}

\begin{tcolorbox}[colback=blue!5!white,colframe=blue!75!black,title=Important]
Vous n'avez PAS besoin de Mac! EAS peut gérer la signature des apps dans le cloud automatiquement.
\end{tcolorbox}

\subsubsection{Configuration iOS dans app.json}

Vérifiez que \texttt{app.json} contient:

\begin{lstlisting}[language=json]
{
  "expo": {
    "ios": {
      "bundleIdentifier": "com.pipomarket.app",
      "buildNumber": "1"
    }
  }
}
\end{lstlisting}

\subsubsection{Lancer le build iOS}

\begin{lstlisting}[language=bash]
eas build --platform ios --profile production
\end{lstlisting}

\textbf{EAS va vous demander:}
\begin{itemize}
    \item "Generate a new Apple Distribution Certificate?": Choisissez \textbf{Yes}
    \item "Generate a new Apple Provisioning Profile?": Choisissez \textbf{Yes}
\end{itemize}

EAS va gérer automatiquement:
\begin{itemize}
    \item Création des certificats
    \item Signature de l'app
    \item Upload vers Apple
\end{itemize}

⏱️ \textbf{Durée}: 20-30 minutes

\subsubsection{Pendant l'attente: Préparer App Store Connect}

\begin{enumerate}
    \item Allez sur: \url{https://appstoreconnect.apple.com}
    \item Cliquez "Mes apps" → "+" → "Nouvelle app"
    \item Remplissez:
    \begin{itemize}
        \item Plateformes: iOS
        \item Nom: PipoMarket
        \item Langue principale: Français
        \item Bundle ID: com.pipomarket.app (doit correspondre à app.json)
        \item SKU: pipomarket (identifiant unique)
    \end{itemize}
\end{enumerate}

\subsection{Préparer la fiche App Store}

\subsubsection{Assets nécessaires}

\textbf{1. Screenshots iPhone (OBLIGATOIRE)}

Vous avez besoin de 2 tailles minimum:

\begin{itemize}
    \item \textbf{iPhone 6.7" (iPhone 15 Pro Max, 14 Pro Max)}
    \begin{itemize}
        \item Dimensions: \textbf{1290 × 2796 px}
        \item Minimum: 3 screenshots
    \end{itemize}
    \item \textbf{iPhone 6.5" (iPhone 11 Pro Max, XS Max)}
    \begin{itemize}
        \item Dimensions: \textbf{1242 × 2688 px}
        \item Minimum: 3 screenshots
    \end{itemize}
\end{itemize}

\textbf{Comment les créer:}
\begin{itemize}
    \item Utilisez le simulateur iOS (si vous avez un Mac)
    \item Ou redimensionnez vos screenshots Android avec Photoshop/GIMP
    \item Ou utilisez un outil comme \url{https://www.mokup.ai/}
\end{itemize}

\textbf{2. Screenshots iPad (OPTIONNEL)}
\begin{itemize}
    \item Dimensions: \textbf{2048 × 2732 px}
\end{itemize}

\textbf{3. Icône app}
\begin{itemize}
    \item Déjà prêt: \texttt{assets/images/icon.png} (1024×1024)
\end{itemize}

\subsubsection{Textes à préparer}

\textbf{Nom de l'app (30 caractères max)}

\textit{PipoMarket}

\textbf{Sous-titre (30 caractères max)}

\textit{Marketplace Camerounaise}

\textbf{Mots-clés (100 caractères max)}

\textit{marketplace,cameroun,startup,shopping,local,cameroon,artisan,entrepreneur}

\textbf{Catégorie primaire}: Shopping

\textbf{Catégorie secondaire}: Business

\subsection{Soumission sur App Store}

\subsubsection{Upload du build}

Quand EAS a terminé:
\begin{enumerate}
    \item Allez sur: \url{https://expo.dev/accounts/[compte]/projects/pipomarket/builds}
    \item Le build iOS apparaît automatiquement dans App Store Connect après 10-15 minutes
    \item Rafraîchissez App Store Connect
\end{enumerate}

\subsubsection{Remplir la fiche App Store Connect}

\begin{enumerate}
    \item \textbf{Informations sur l'app:}
    \begin{itemize}
        \item Uploadez tous les screenshots
        \item Description, sous-titre, mots-clés
        \item URL de support et confidentialité
    \end{itemize}
    \item \textbf{Classification du contenu:}
    \begin{itemize}
        \item Âge: 4+ (tout public)
        \item Pas de contenu sensible
    \end{itemize}
    \item \textbf{Informations sur les prix:}
    \begin{itemize}
        \item Gratuite
        \item Disponible dans tous les pays
    \end{itemize}
    \item \textbf{Préparation pour la soumission:}
    \begin{itemize}
        \item Section "Informations sur la version"
        \item Nouveautés: "Première version de PipoMarket"
        \item Copyright: "2025 PipoMarket"
    \end{itemize}
    \item Cliquez \textbf{"Envoyer pour examen"}
\end{enumerate}

\subsubsection{Attente de la review}

\begin{itemize}
    \item ⏱️ \textbf{1 à 2 semaines} en moyenne (parfois moins)
    \item Statut visible dans App Store Connect
    \item Apple peut demander des clarifications (répondez rapidement)
\end{itemize}

\subsection{Problèmes courants}

\subsubsection{Build échoue: Invalid bundle identifier}

\textbf{Solution:} Vérifiez que \texttt{bundleIdentifier} (iOS) et \texttt{package} (Android) dans \texttt{app.json} sont corrects et uniques.

\subsubsection{Screenshots mauvaises dimensions}

\textbf{Solution:} Utilisez exactement les dimensions requises. Pas d'approximation!

\subsubsection{Missing privacy policy}

\textbf{Solution:} Vous DEVEZ avoir une URL accessible publiquement avec votre politique de confidentialité.

\subsubsection{Build très long (>1h)}

\textbf{Solution:} Normal parfois. Vérifiez sur expo.dev que le build n'a pas échoué.

\newpage

% =====================================================
% PARTIE 3: TEMPLATES
% =====================================================

\section{Templates - Textes Prêts à Copier}

\subsection{Play Store}

\subsubsection{Description courte (80 caractères max)}

\begin{tcolorbox}[colback=green!5!white,colframe=green!75!black]
Marketplace camerounaise - Découvrez et soutenez les startups locales
\end{tcolorbox}

\subsubsection{Description complète}

\begin{tcolorbox}[colback=green!5!white,colframe=green!75!black]
\small
🛍️ PipoMarket - La marketplace qui met en valeur les startups camerounaises

Découvrez des produits uniques créés par des entrepreneurs locaux passionnés. PipoMarket connecte les startups camerounaises avec leurs clients, facilitant l'achat et la promotion de produits locaux.

✨ FONCTIONNALITÉS

🏪 Pour les acheteurs:
• Parcourez des centaines de produits de startups locales
• Commandez facilement avec livraison à domicile
• Payez via Mobile Money (Orange Money, MTN)
• Suivez vos commandes en temps réel
• Découvrez de nouvelles startups chaque jour

🚀 Pour les startups:
• Créez votre boutique en quelques minutes
• Gérez vos produits et stocks
• Recevez des commandes instantanément
• 3 formules d'abonnement adaptées à vos besoins
• Dashboard complet pour suivre vos ventes

💳 PAIEMENT SIMPLE

Paiement sécurisé via Mobile Money:
• Orange Money
• MTN Mobile Money

📦 CATÉGORIES

• Mode \& Accessoires
• Alimentation \& Boissons
• Technologie
• Artisanat
• Beauté \& Cosmétiques
• Et bien plus...

🇨🇲 100\% CAMEROUNAIS

PipoMarket est fier de soutenir l'entrepreneuriat local. Chaque achat aide une startup camerounaise à grandir.

Téléchargez maintenant et rejoignez le mouvement \#ConsommerLocal 🇨🇲🔥

---

📧 Support: support@pipomarket.com

🌐 Site web: https://pipomarket.com
\end{tcolorbox}

\subsection{App Store}

\subsubsection{Nom de l'app (30 caractères max)}

\begin{tcolorbox}[colback=blue!5!white,colframe=blue!75!black]
PipoMarket
\end{tcolorbox}

\subsubsection{Sous-titre (30 caractères max)}

\begin{tcolorbox}[colback=blue!5!white,colframe=blue!75!black]
Marketplace Camerounaise
\end{tcolorbox}

\subsubsection{Texte promotionnel (170 caractères max)}

\begin{tcolorbox}[colback=blue!5!white,colframe=blue!75!black]
Découvrez et soutenez les startups camerounaises. Achetez local, payez via Mobile Money. Rejoignez le mouvement \#ConsommerLocal 🇨🇲
\end{tcolorbox}

\subsubsection{Mots-clés (100 caractères max)}

\begin{tcolorbox}[colback=blue!5!white,colframe=blue!75!black]
marketplace,cameroun,startup,shopping,local,cameroon,artisan,entrepreneur,mobile money
\end{tcolorbox}

\subsubsection{Nouveautés de cette version}

\begin{tcolorbox}[colback=blue!5!white,colframe=blue!75!black]
🎉 Première version de PipoMarket!

• Marketplace complète pour startups camerounaises
• Paiement Mobile Money (Orange, MTN)
• Gestion de boutique pour entrepreneurs
• Système de commandes et livraison
• 3 formules d'abonnement startup

Bienvenue sur PipoMarket! 🇨🇲
\end{tcolorbox}

\subsection{Politique de confidentialité}

\subsubsection{Template complet}

\begin{tcolorbox}[colback=yellow!5!white,colframe=yellow!75!black,title=À mettre sur pipomarket.com/privacy]
\scriptsize

\textbf{POLITIQUE DE CONFIDENTIALITÉ - PIPOMARKET}

\textbf{Dernière mise à jour:} [Insère la date d'aujourd'hui]

Chez PipoMarket, nous respectons votre vie privée et nous nous engageons à protéger vos données personnelles.

\textbf{1. Informations que nous collectons}

Informations fournies par vous:
• Compte utilisateur: Nom, prénom, adresse email, numéro de téléphone
• Profil startup: Nom de l'entreprise, description, catégorie, logo
• Commandes: Adresse de livraison, historique d'achats
• Paiements: Informations de transaction Mobile Money

Informations collectées automatiquement:
• Données d'utilisation: Pages visitées, interactions
• Données techniques: Type d'appareil, système d'exploitation, IP

\textbf{2. Comment nous utilisons vos informations}

• Traiter vos commandes et gérer les livraisons
• Vous contacter concernant vos achats
• Améliorer nos services
• Prévenir la fraude
• Envoyer des notifications (avec consentement)

\textbf{3. Partage de vos informations}

Nous ne vendons jamais vos données. Nous les partageons uniquement avec:
• Les startups vendeurs (pour traiter commandes)
• Services de paiement (Orange Money, MTN)
• Prestataires techniques (Firebase/Google Cloud)
• Autorités légales (si requis par la loi)

\textbf{4. Sécurité de vos données}

• Stockage sécurisé sur Firebase (Google Cloud)
• Connexions chiffrées (HTTPS/SSL)
• Accès restreint et contrôlé

\textbf{5. Conservation des données}

• Compte actif: Tant que votre compte existe
• Historique commandes: 5 ans
• Données paiement: Durée minimale requise

\textbf{6. Vos droits}

Vous avez le droit de:
• Accéder à vos données
• Rectifier vos informations
• Supprimer votre compte
• Exporter vos données
• Retirer votre consentement

Pour exercer ces droits: privacy@pipomarket.com

\textbf{7. Contact}

📧 Email: privacy@pipomarket.com

📧 Support: support@pipomarket.com

🌐 Site web: https://pipomarket.com

© 2025 PipoMarket. Tous droits réservés.
\end{tcolorbox}

\subsection{Informations de contact}

\begin{table}[h]
\centering
\begin{tabular}{|l|l|}
\hline
\textbf{Type} & \textbf{Valeur} \\
\hline
Email support & support@pipomarket.com \\
Email privacy & privacy@pipomarket.com \\
Site web & https://pipomarket.com \\
URL privacy policy & https://pipomarket.com/privacy \\
URL support & https://pipomarket.com/support \\
Copyright & © 2025 PipoMarket \\
\hline
\end{tabular}
\caption{Informations de contact pour les stores}
\end{table}

\subsection{Informations techniques}

\subsubsection{Android (Play Store)}

\begin{table}[h]
\centering
\begin{tabular}{|l|l|}
\hline
\textbf{Champ} & \textbf{Valeur} \\
\hline
Package name & com.pipomarket.app \\
Version & 1.0.0 \\
Version code & 1 \\
Minimum SDK & 21 (Android 5.0) \\
Target SDK & 34 (Android 14) \\
\hline
\end{tabular}
\caption{Informations techniques Android}
\end{table}

\subsubsection{iOS (App Store)}

\begin{table}[h]
\centering
\begin{tabular}{|l|l|}
\hline
\textbf{Champ} & \textbf{Valeur} \\
\hline
Bundle Identifier & com.pipomarket.app \\
Version & 1.0.0 \\
Build Number & 1 \\
Minimum iOS version & 13.0 \\
\hline
\end{tabular}
\caption{Informations techniques iOS}
\end{table}

\newpage

% =====================================================
% PARTIE 4: AIDE ET SUPPORT
% =====================================================

\section{Aide et Support}

\subsection{Ressources utiles}

\begin{itemize}
    \item \textbf{Expo Dashboard:} \url{https://expo.dev}
    \item \textbf{Play Console:} \url{https://play.google.com/console}
    \item \textbf{App Store Connect:} \url{https://appstoreconnect.apple.com}
    \item \textbf{Documentation EAS:} \url{https://docs.expo.dev/build/introduction/}
    \item \textbf{Play Console Help:} \url{https://support.google.com/googleplay/android-developer/}
    \item \textbf{App Store Help:} \url{https://developer.apple.com/support/app-store-connect/}
\end{itemize}

\subsection{Timeline réaliste}

\begin{table}[h]
\centering
\begin{tabular}{|l|l|}
\hline
\textbf{Étape} & \textbf{Temps} \\
\hline
Créer comptes développeurs & 30 min \\
Préparer assets (screenshots, textes) & 2-3 heures \\
Build Android & 30 min \\
Remplir fiche Play Store & 1 heure \\
Build iOS & 30 min \\
Remplir fiche App Store & 1 heure \\
\hline
\textbf{TOTAL travail actif} & \textbf{~6 heures} \\
\hline
\hline
Review Play Store & 3-7 jours \\
Review App Store & 1-2 semaines \\
\hline
\textbf{TOTAL jusqu'à publication} & \textbf{1-2 semaines} \\
\hline
\end{tabular}
\caption{Timing réaliste complet}
\end{table}

\subsection{Commandes récapitulatives}

Voici TOUTES les commandes dans l'ordre:

\begin{lstlisting}[language=bash,caption={Commandes complètes pour publication}]
# 1. Installation EAS CLI
npm install -g eas-cli

# 2. Connexion a Expo
eas login

# 3. Initialisation du projet
eas init

# 4. Build Android
eas build --platform android --profile production

# 5. Build iOS (apres Android lance)
eas build --platform ios --profile production

# 6. Verifier les builds
# Allez sur: https://expo.dev
\end{lstlisting}

\subsection{Conseils finaux}

\begin{enumerate}
    \item \textbf{Faites Android d'abord} - Plus simple, review plus rapide
    \item \textbf{Testez TOUT avant de soumettre} - Une rejection retarde de plusieurs jours
    \item \textbf{Préparez tous les assets AVANT} de commencer les builds
    \item \textbf{Répondez vite} aux questions de Google/Apple
    \item \textbf{Screenshots = super important} - Montrez les meilleures fonctionnalités
    \item \textbf{Description = vente} - Expliquez pourquoi télécharger votre app
\end{enumerate}

\newpage

% =====================================================
% CONCLUSION
% =====================================================

\section{Conclusion}

\subsection{Récapitulatif}

\begin{tcolorbox}[colback=green!10!white,colframe=green!75!black,title=🎯 Objectif]
Publier PipoMarket sur Play Store et App Store en 1-2 semaines
\end{tcolorbox}

\textbf{Budget total:} 124\$ (25\$ + 99\$)

\textbf{Temps de travail:} ~6 heures

\textbf{Résultat:} Application disponible mondialement sur les deux principales plateformes mobiles

\subsection{Prochaines étapes}

\begin{enumerate}
    \item \textbf{Aujourd'hui}: Lire ce guide en entier
    \item \textbf{Demain matin}: Créer les comptes développeurs
    \item \textbf{Demain midi}: Préparer les assets
    \item \textbf{Demain après-midi}: Lancer les builds et soumettre
    \item \textbf{Dans 1-2 semaines}: Applications en ligne!
\end{enumerate}

\begin{center}
\Large\textbf{Bonne chance! 🚀}

\vspace{0.5cm}

\textit{PipoMarket - Made in Cameroon 🇨🇲}
\end{center}

\end{document}
